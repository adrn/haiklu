% TODO list
% ----------
% -

% Style rules
% -----------
% - Don't be typing ``pdflatex'' or some bullshit; use the Makefile.
% - 80 character hard wrapping.
% - Line break between sentences to make the git diffs readable.
% - Use \, as a multiply operator.
% - Use \sectionname not Section, \figname not Figure, \documentname not
%   Article or Paper or paper.

\documentclass[modern, letterpaper]{aastex61}

\include{gitstuff}
% Load common packages

\usepackage{amsmath}
\usepackage{amsfonts}
\usepackage{amssymb}
\usepackage{booktabs}

\usepackage{graphicx}
\usepackage{color}

\usepackage{hyperref}
\definecolor{cbblue}{HTML}{3182bd}
\hypersetup{colorlinks=true,linkcolor=cbblue,citecolor=cbblue,
            filecolor=cbblue,urlcolor=cbblue}
\hypersetup{pageanchor=false}

\newcommand{\documentname}{\textsl{Article}}
\newcommand{\sectionname}{Section}
\newcommand{\figname}{Figure}
\newcommand{\eqname}{Equation}
\newcommand{\tblname}{Table}

% Packages / projects / programming
\newcommand{\package}[1]{\textsl{#1}}
\newcommand{\acronym}[1]{{\small{#1}}}
\newcommand{\project}[1]{\package{#1}}
\newcommand{\github}{\project{GitHub}}
\newcommand{\python}{\project{Python}}

% For referee
\definecolor{mahogany}{RGB}{165,15,21}
\newcommand{\resp}[1]{{\color{mahogany}#1}}

% Stats / probability
\newcommand{\given}{\,|\,}
\newcommand{\norm}{\mathcal{N}}

% Maths
\newcommand{\dd}{\mathrm{d}}
\newcommand{\transpose}[1]{{#1}^{\mathsf{T}}}
\newcommand{\inverse}[1]{{#1}^{-1}}
\newcommand{\argmin}{\operatornamewithlimits{argmin}}
\newcommand{\mean}[1]{\left< #1 \right>}

% Unit shortcuts
\newcommand{\msun}{\ensuremath{\mathrm{M}_\odot}}
\newcommand{\kms}{\ensuremath{\mathrm{km}~\mathrm{s}^{-1}}}
\newcommand{\pc}{\ensuremath{\mathrm{pc}}}
\newcommand{\kpc}{\ensuremath{\mathrm{kpc}}}
\newcommand{\kmskpc}{\ensuremath{\mathrm{km}~\mathrm{s}^{-1}~\mathrm{kpc}^{-1}}}

% Misc.
\newcommand{\bs}[1]{\boldsymbol{#1}}

% Astronomy
\newcommand{\DM}{{\rm DM}}
\newcommand{\feh}{\ensuremath{{[{\rm Fe}/{\rm H}]}}}
\newcommand{\df}{\acronym{DF}}

% TO DO
\newcommand{\todo}[1]{{\color{red} TODO: #1}}


% packages
\usepackage{microtype}  % ALWAYS!

% Macros for text - most are defined in preamble.tex
% \newcommand{\apogee}{\project{\acronym{APOGEE}}}
% \newcommand{\sdssiii}{\project{\acronym{SDSS-III}}}

\begin{document}\sloppy\sloppypar\raggedbottom\frenchspacing % trust me

\title{Kinematic and chemical tagging of co-eval clusters of co-eval stars}

\author{Adrian~M.~Price-Whelan}
\altaffiliation{To whom correspondence should be addressed:
                \texttt{adrn@astro.princeton.edu}}
\affiliation{Department of Astrophysical Sciences,
             Princeton University, Princeton, NJ 08544, USA}

\author{David~W.~Hogg}
\affiliation{Center for Cosmology and Particle Physics,
             Department of Physics,
             New York University, 4 Washington Place,
             New York, NY 10003, USA}
\affiliation{Center for Data Science,
                New York University, 60 Fifth Avenue,
                New York, NY 10011, USA}
\affiliation{Max-Planck-Institut f\"ur Astronomie,
                K\"onigstuhl 17, D-69117 Heidelberg, Germany}
\affiliation{Flatiron Institute,
                Simons Foundation, 162 Fifth Avenue,
                New York, NY 10010, USA}

\begin{abstract}
% Context
The hope of chemical tagging is to find widely separated but originally
co-eval stars through the similarity of their detailed chemical abundances;
the high-dimensional chemical abundance is expected to be a cryptographic
hash of the stellar birth cloud and time.
Star formation is a hierarchical process, with stars forming in
groups, which themselves form in larger clusters, which themselves
form in larger star-formation regions.
Therefore chemical tagging ought to be applicable to star clusters
just as it is to stars.
% Aims
Here we show that it is possible to perform far more precise chemical
tagging using clusters or associations of stars than it is with any
individual star.
The precision comes not just from the numbers, but from the fact that
stars in a cluster span stellar parameter space at fixed chemical
abundances; this makes comparison of clusters in abundance space far
less model-dependent than the equivalent comparison of stars.
% Methods
We compare the spectra as a function of effective temperature for
red-giant stars from open clusters XXX, YYY, and ZZZ.
Following previous work, we build a smooth, data-driven model for the
spectra in each cluster as a function of a single latent parameter
(mass or effective temperature).
% Results
We find AAA and BBB, and can rule out CCC and DDD.
The principal limitation of this work is that open clusters are not
long-lived; so in the disk these techniques are restricted to stars of
recent vintage.
\end{abstract}

\keywords{
}

\section{Introduction} \label{sec:intro}

\software{The code used in this project is available from
\url{https://github.com/adrn/haiklu} under the MIT open-source software
license. This version was generated at git commit
\texttt{\githash\,(\gitdate)}.
This research additionally utilized:
    \texttt{Astropy} (\citealt{Astropy-Collaboration:2013}),
    \texttt{IPython} (\citealt{Perez:2007}),
    \texttt{matplotlib} (\citealt{Hunter:2007}),
    and \texttt{numpy} (\citealt{Van-der-Walt:2011}).}

% \facility{\sdssiii, \apogee}

\bibliographystyle{aasjournal}
% \bibliography{thejoker}

\end{document}
